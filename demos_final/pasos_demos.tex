\documentclass[12pt,a4paper]{article}
\input{packages.tex}
\author{Mansilla, Kevin Gaston\footnote{\href{mailto:kevingston47@gmail.com}{kevingston47@gmail.com}}}
\title{Discreta II: Pasos para las demos}
\date{\today}
\begin{document}
\maketitle{}

%%%%%% 1 %%%%%%
\begin{itemize}
    \item [1)] \textbf{Cual es la complejidad del algoritmo de Edmonds-Karp? Probarlo (
        Nota: en la prueba se definen unas distancias, y se prueba que esas 
        distancias no disminuyen en pasos sucesivos de EK. Ud. puede usar esto 
        sin necesidad de probarlo.)}
        \label{dem:EK}
\end{itemize}

%%%%%% 2 %%%%%%
\begin{itemize}
    \item [2)] \textbf{Probar que si, dados vértices $x$, $z$ y flujo $f$ definimos a la distancia 
    entre $x$ y $z$ relativa a $f$ como la longitud del menor $f-$camino aumentante 
    entre $x$ y $z$, si es que existe tal camino, o infinito si no existe o 0 
    si $x = z$, denotandola por $d_{f} (x, z)$, y definimos 
    $d_{k}(x) = d_{f_{k}} (s, x)$, donde $f_{k}$ es el $k-$ésimo flujo en una 
    corrida de Edmonds-Karp, entonces $d_{k}(x) \< d_{k+1}(x)$}
    \label{dem:dist}
\end{itemize}

%%%%%% 3 %%%%%%
\begin{itemize}
    \item [3)] \textbf{Cual es la complejidad del algoritmo de Dinic? Probarla en ambas 
    versiones: Dinitz original y Dinic-Even. (no hace falta probar que la distancia 
    en networks auxiliares sucesivos aumenta)}
    \label{dem:Dinic}
\end{itemize}

%%%%%% 4 %%%%%%
\begin{itemize}
    \item [4)] \textbf{Cual es la complejidad del algoritmo de Wave? Probarla. (no hace falta 
    probar que la distancia en networks auxiliares sucesivos aumenta).}
    \label{dem:Wave}
\end{itemize}

%%%%%% 5 %%%%%%
\begin{itemize}
    \item [5)] \textbf{Probar que la distancia en networks auxiliares sucesivos aumenta.}
    \label{dem:dist2}
\end{itemize}

%%%%%% 6 %%%%%%
\begin{itemize}
    \item [6)] \textbf{Si $f$ es flujo las siguientes son equivalentes:
    \begin{itemize}
        \item [1.] $\exists S$ corte: $v(f) = cap(S)$
        \item [2.] $f$ es maximal.\\
        $(1=2)$ dice: \\
        "$f$ maximal $\Longleftrightarrow \exists S$ corte $v(f) = cap(S)$"\\
        y se suele llamar 'max-flow-min-cut theorem'.
        \item [3.] $\nexists f-$caminos aumentanes entre $s$ y $t$\\
        y si se cumplen, el $s$ es minimal.
    \end{itemize}}
\end{itemize}

%%%%%% 7 %%%%%%
\begin{itemize}
    \item [7)] \textbf{Probar que $2-$COLOR es polinomial.}
    \label{dem:2color}
\end{itemize}

%%%%%% 8 %%%%%%
\begin{itemize}
    \item [8)] \textbf{Enunciar y probar el Teorema de Hall.}
    \label{dem:hall}
\end{itemize}

%%%%%% 9 %%%%%%
\begin{itemize}
    \item [9)] \textbf{Enunciar y probar el teorema del matrimonio de Konig}
    \label{dem:konig}
\end{itemize}

%%%%%% 10 %%%%%%
\begin{itemize}
    \item [10)] \textbf{Probar que si $G$ es bipartito entonces $\chi^{'}(G) = \Delta(G)$}
    \label{dem:chi}
\end{itemize}

%%%%%% 11 %%%%%%
\begin{itemize}
    \item [11)] \textbf{Probar la complejidad $O(n^{4})$ del algoritmo Hungaro y dar una idea 
    de como se la puede reducir a $O(n^{3})$.}
    \label{dem:hungaro}
\end{itemize}


%%%%%% 12 %%%%%%
\begin{itemize}
    \item [12)] \textbf{Enunciar el teorema de la cota de Hamming y probarlo}
    \label{dem:hamming}
\end{itemize}

%%%%%% 13 %%%%%%
\begin{itemize}
    \item [13)] \textbf{Probar que si $H$ es matriz de chequeo de $C$, entonces
        $$\delta(C) = \min {j: \exists \,\,\text{un conjunto de $j$ columnas LD de $H$}}$$
        (LD es linealmente dependiente)}
    \label{dem:delta}
\end{itemize}

%%%%%% 14 %%%%%%
\begin{itemize} 
    \item [14)]\textbf{(Fundamental de código ciclico) Sea $C$ un código ciclico de 
    longitud $n$ con generador $g(x)$ entonces:
    \begin{itemize}
        \item [1)] $C = \llaves{p(x)\in \mathbb{Z}_{2}(x):gr(p)<n \wedge g(x)|p(x)}$ por esto 
            se dice que $C$ es generador (son los multiplicos de $g(x)$ de menor grado).
        \item [2)] $C = \llaves{v(x)\odot g(x): v\in \mathbb{Z}_{2}(x)}$ son los multiplos 
            de $g$ modulares.
        \item [3)] Si $k=Dim(C)$ entonces $gr(g)=n-k$.
        \item [4)] $g(x) | (1+x^{n})$.
        \item [5)] Si $g(x) = g_{0} + g_{1}x + \ldots +$ entonces $g_{0}=1$.
    \end{itemize}}
\end{itemize}

%%%%%% 15 %%%%%%
\begin{itemize}
    \item [15)] \textbf{Probar que $3$SAT es NP-completo}
    \label{dem:3sat}
\end{itemize}


%%%%%% 16 %%%%%%
\begin{itemize}
    \item [16)] \textbf{Probar que $3-$COLOR es NP-completo}
    \label{dem:3color}
\end{itemize}

\end{document}