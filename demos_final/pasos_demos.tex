\documentclass[12pt,a4paper]{article}
\input{packages.tex}
\begin{document}
\maketitle{}

%%%%%% 1 %%%%%%
\begin{itemize}
    \item [1)] \textbf{Cual es la complejidad del algoritmo de Edmonds-Karp? Probarlo (
        Nota: en la prueba se definen unas distancias, y se prueba que esas 
        distancias no disminuyen en pasos sucesivos de EK. Ud. puede usar esto 
        sin necesidad de probarlo.)}
        \label{dem:EK}
\end{itemize}

%%%%%% 2 %%%%%%
\begin{itemize}
    \item [2)] \textbf{Probar que si, dados vértices $x$, $z$ y flujo $f$ definimos a la distancia 
    entre $x$ y $z$ relativa a $f$ como la longitud del menor $f-$camino aumentante 
    entre $x$ y $z$, si es que existe tal camino, o infinito si no existe o 0 
    si $x = z$, denotandola por $d_{f} (x, z)$, y definimos 
    $d_{k}(x) = d_{f_{k}} (s, x)$, donde $f_{k}$ es el $k-$ésimo flujo en una 
    corrida de Edmonds-Karp, entonces $d_{k}(x) \< d_{k+1}(x)$}
    \label{dem:dist}
\end{itemize}

%%%%%% 3 %%%%%%
\begin{itemize}
    \item [3)] \textbf{Cual es la complejidad del algoritmo de Dinic? Probarla en ambas 
    versiones: Dinitz original y Dinic-Even. (no hace falta probar que la distancia 
    en networks auxiliares sucesivos aumenta)}
    \label{dem:Dinic}
\end{itemize}

%%%%%% 4 %%%%%%
\begin{itemize}
    \item [4)] \textbf{Cual es la complejidad del algoritmo de Wave? Probarla. (no hace falta 
    probar que la distancia en networks auxiliares sucesivos aumenta).}
    \label{dem:Wave}
\end{itemize}





\end{document}